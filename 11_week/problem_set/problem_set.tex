\documentclass[11pt]{article}
\usepackage{amsmath, amssymb, graphicx, hyperref}
\usepackage{enumitem}
\setlist{nosep}
\usepackage[margin=1in]{geometry}

\title{In-Class Problem Set: Scatterplots and Association (R + GitHub)}
\author{}
\date{}

\begin{document}
\maketitle

\noindent \textbf{Goal.} Use NBA player data to practice visualizing relationships between \textbf{two quantitative variables} using scatterplots, smoothers, polynomial fits, and careful interpretation. You will pull the data from GitHub, clean variable types, generate required figures, interpret what they show, and submit via GitHub.

\medskip
\noindent \textbf{Dataset.} \texttt{basketball} (569 rows; 23 variables). Key variables you may use:
\begin{itemize}
  \item IDs and labels: \texttt{PLAYER\_NAME}, \texttt{TEAM\_ABBREVIATION}
  \item Player attributes: \texttt{AGE}, \texttt{PLAYER\_HEIGHT\_INCHES}, \texttt{PLAYER\_WEIGHT}
  \item Games + performance: \texttt{GP}, \texttt{PTS}, \texttt{REB}, \texttt{AST}, \texttt{NET\_RATING}
  \item Rates (0--1): \texttt{OREB\_PCT}, \texttt{DREB\_PCT}, \texttt{USG\_PCT}, \texttt{TS\_PCT}, \texttt{AST\_PCT}
  \item Draft info: \texttt{DRAFT\_YEAR} (contains ``Undrafted'' for some players)
\end{itemize}

\medskip
\noindent \textbf{Important data note.} Many columns are stored as \textbf{character strings}. You must convert variables you analyze into appropriate numeric types before plotting.

\medskip
\noindent \textbf{What to submit (in your GitHub repo).}
\begin{itemize}
  \item A script file: \texttt{scripts/lab.R}
  \item A short write-up: \texttt{outputs/writeup.md}
  \item Saved figures in \texttt{figures/} (see requirements below)
\end{itemize}

\medskip
\noindent \textbf{Rules.}
\begin{itemize}
  \item Work inside an \textbf{R Project}.
  \item Use a \textbf{sequential, hard-coded workflow} (no user-defined functions).
  \item Save figures using \texttt{ggsave()} (no screenshots).
  \item Git commands must be run in the \textbf{Terminal tab}, not the R Console.
  \item Use \texttt{theme\_classic()} unless you explicitly justify an alternative.
  \item Handle missing values defensibly (state what you did).
\end{itemize}

\section*{Questions}

\begin{enumerate}

  \item \textbf{Pull the repo and confirm the dataset (proof required).}
  \begin{enumerate}
    \item In the \textbf{Terminal tab}, run:
\begin{verbatim}
git status
git pull
\end{verbatim}
    \item Confirm the dataset file exists in your repo (path posted in the course repository).
    \item Create the standard folder structure (if missing): \texttt{scripts/}, \texttt{outputs/}, \texttt{figures/}.
    \item \textbf{Proof (write-up):} In \texttt{outputs/writeup.md}, paste:
    \begin{itemize}
      \item the output of \texttt{getwd()},
      \item the output of \texttt{list.files()} from the project root, and
      \item the output of \texttt{list.files("data")} showing the dataset file.
    \end{itemize}
  \end{enumerate}

  \item \textbf{Load and clean \texttt{basketball} (proof required).}
  \begin{enumerate}
    \item Load the dataset into an object named \texttt{basketball}.
    \item Create a cleaned object named \texttt{basketball\_clean} where you convert the following columns to numeric:
    \[
      \texttt{AGE},\ \texttt{PLAYER\_HEIGHT\_INCHES},\ \texttt{PLAYER\_WEIGHT},\ \texttt{GP},\ \texttt{PTS},\ \texttt{REB},\ \texttt{AST},\ \texttt{NET\_RATING},
    \]
    \[
      \texttt{OREB\_PCT},\ \texttt{DREB\_PCT},\ \texttt{USG\_PCT},\ \texttt{TS\_PCT},\ \texttt{AST\_PCT}.
    \]
    \item Create a simple draft indicator:
    \[
      \texttt{draft\_status} = \text{``Undrafted'' vs ``Drafted''}
    \]
    where ``Drafted'' means \texttt{DRAFT\_YEAR} is not ``Undrafted''.
    \item \textbf{Proof (write-up):} Report:
    \begin{itemize}
      \item the number of rows in \texttt{basketball} and \texttt{basketball\_clean},
      \item a quick summary of at least three numeric columns (e.g., \texttt{AGE}, \texttt{PTS}, \texttt{USG\_PCT}),
      \item one sentence describing how you handled missing or non-numeric values after conversion.
    \end{itemize}
  \end{enumerate}

  \item \textbf{Relationship 1: usage and scoring (scatterplot baseline).}

  Make a scatterplot of \texttt{USG\_PCT} (x) vs \texttt{PTS} (y).
  \begin{itemize}
    \item Use \texttt{geom\_point()} with an overplotting fix (e.g., \texttt{alpha} and/or smaller \texttt{size}).
    \item Use clear labels. Since \texttt{USG\_PCT} is a proportion, format the x-axis as percent if you can.
    \item Use \texttt{theme\_classic()}.
  \end{itemize}

  \noindent Save as:
  \[
    \texttt{figures/usg\_pts\_scatter.png}
  \]

  \item \textbf{Relationship 1 (extension): add a linear smoother with standard errors.}

  Using the same x/y pairing as the previous question, add a linear fit:
  \begin{itemize}
    \item \texttt{geom\_smooth(method = "lm", se = TRUE)}
    \item Keep the points visible (do not remove them).
  \end{itemize}

  \noindent \textbf{Write-up (3--5 sentences):}
  \begin{itemize}
    \item What does the fitted line claim?
    \item What does the shaded band represent (in plain language)?
    \item Does the band make you more or less confident about the trend?
  \end{itemize}

  \noindent Save as:
  \[
    \texttt{figures/usg\_pts\_lm\_se.png}
  \]

  
  \item \textbf{Relationship 1B: usage and scoring efficiency.}

  Make a scatterplot of \texttt{USG\_PCT} (x) vs \texttt{TS\_PCT} (y).
  \begin{itemize}
    \item Use \texttt{geom\_point()} with an overplotting fix (e.g., \texttt{alpha} and/or smaller \texttt{size}).
    \item Format \texttt{USG\_PCT} and \texttt{TS\_PCT} as percents if you can.
    \item Add a linear smoother with standard errors: \texttt{geom\_smooth(method = "lm", se = TRUE)}.
    \item Use \texttt{theme\_classic()}.
  \end{itemize}

  \noindent \textbf{Write-up (3--5 sentences):}
  \begin{itemize}
    \item Does the trend look linear, or do you suspect curvature?
    \item What does the SE ribbon suggest about uncertainty across usage levels?
  \end{itemize}

  \noindent Save as:
  \[
    \texttt{figures/usg\_ts\_eff\_lm\_se.png}
  \]

  \item \textbf{Relationship 1C: assists (raw vs rate).}

  Make a scatterplot of \texttt{AST} (x) vs \texttt{AST\_PCT} (y).
  \begin{itemize}
    \item Use \texttt{geom\_point()} with an overplotting fix (e.g., \texttt{alpha} and/or smaller \texttt{size}).
    \item Format \texttt{AST\_PCT} as a percent if you can.
    \item Add a linear smoother with standard errors: \texttt{geom\_smooth(method = "lm", se = TRUE)}.
    \item Use \texttt{theme\_classic()}.
  \end{itemize}

  \noindent \textbf{Write-up (3--5 sentences):}
  \begin{itemize}
    \item Are \texttt{AST} and \texttt{AST\_PCT} close to a one-to-one relationship, or are there notable exceptions?
    \item Give one plausible basketball reason you might see players with similar \texttt{AST} but different \texttt{AST\_PCT}.
  \end{itemize}

  \noindent Save as:
  \[
    \texttt{figures/ast\_astpct\_lm\_se.png}
  \]

  \item \textbf{Final interpretation (write-up required).}

  In \texttt{outputs/writeup.md}, write 12--16 sentences addressing:
  \begin{itemize}
    \item For each figure, what is the main pattern (direction + strength + shape)?
    \item Name one outlier or ``surprising'' point pattern and what it could imply (without claiming causality).
    \item Name one concrete plotting choice you made (alpha, axis formatting, legend placement, polynomial fit) and why it helped interpretability.
  \end{itemize}

  \item \textbf{Git workflow and submission (proof required).}

  You must show evidence of both \textbf{pull} and \textbf{push}, plus at least two commits.

  \begin{enumerate}
    \item After you finish cleaning the data (Question 2), commit and push:
\begin{verbatim}
git status
git add .
git commit -m "NBA relationships: clean basketball  types"
git push
\end{verbatim}

    \item Before your final push, run a fresh pull (to catch updates):
\begin{verbatim}
git pull
\end{verbatim}

    \item Commit and push your figures + write-up:
\begin{verbatim}
git status
git add .
git commit -m "NBA relationships: scatterplots + smoothers + writeup"
git push
\end{verbatim}

    \item \textbf{Proof (write-up):} Paste:
    \begin{itemize}
      \item the output of \texttt{git status} after the final push (clean working tree), and
      \item the output of \texttt{git log -2}.
    \end{itemize}
  \end{enumerate}

\end{enumerate}

\section*{Optional challenge (if you finish early)}
Choose one:
\begin{itemize}
  \item Create a small-multiple version of one relationship by faceting on \texttt{TEAM\_ABBREVIATION} for \textbf{only} the teams with at least 12 players in the dataset. (State your rule and why you chose it.)
  \item Create a correlation heatmap for the numeric columns you used (and in 4--6 sentences, explain what the heatmap hides that a scatterplot reveals).
\end{itemize}

\section*{Checklist (before you leave)}
\begin{itemize}
  \item \texttt{scripts/lab.R} runs top-to-bottom
  \item Required figures exist in \texttt{figures/}:
  \begin{itemize}
    \item \texttt{usg\_pts\_scatter.png}, \texttt{usg\_pts\_lm\_se.png}
    \item \texttt{usg\_ts\_eff\_lm\_se.png}, \texttt{ast\_astpct\_lm\_se.png}
    \item \texttt{size\_reb\_scatter.png}
    \item \texttt{age\_ts\_poly2\_se.png}
  \end{itemize}
  \item \texttt{outputs/writeup.md} includes required proofs + interpretation
  \item At least two commits + pushed to GitHub
\end{itemize}

\end{document}
