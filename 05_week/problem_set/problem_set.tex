\documentclass[11pt]{article}
\usepackage{amsmath, amssymb, graphicx, hyperref}
\usepackage{enumitem}
\setlist{nosep}
\usepackage[margin=1in]{geometry}

\title{In-Class Problem Set: Exploring Movie Data with Distribution and Color (R + GitHub \emph{or} Canvas)}
\author{}
\date{}

\begin{document}
\maketitle

\noindent \textbf{Goal.} Use real movie data to practice visualizing distributions, comparing groups, and encoding multiple variables in a single plot. You will obtain the dataset from the course materials (GitHub \emph{or} Canvas), build a reproducible workflow, generate several plots, interpret what they show, and submit your work via \textbf{GitHub \emph{or} Canvas}.

\medskip
\noindent \textbf{What to submit (GitHub \emph{or} Canvas).}
\begin{itemize}
  \item A script file: \texttt{scripts/lab.R}
  \item A short write-up: \texttt{outputs/writeup.md}
  \item Saved figures in \texttt{figures/} (see requirements below)
\end{itemize}

\noindent If you submit via Canvas, upload the same files listed above as individual files
(or upload a single zip that preserves the directory structure).

\medskip
\noindent \textbf{Rules.}
\begin{itemize}
  \item Work inside an \textbf{R Project}.
  \item Use a \textbf{sequential, hard-coded workflow} (no user-defined functions).
  \item Save figures using code (\texttt{ggsave}); do not use screenshots.
  \item If you choose the \textbf{GitHub submission option}, Git commands must be run in the \textbf{Terminal tab}, not the R Console.
\end{itemize}

\section*{Submission options}

You may submit this assignment using \textbf{either} of the following methods:

\begin{itemize}
  \item \textbf{GitHub submission (recommended):}
  Commit and push your work to your GitHub repository. You will include Git proof
  (\texttt{git status} and \texttt{git log}) in your write-up.
  \item \textbf{Canvas submission:}
  Upload the required files directly to Canvas. You do \emph{not} need to use GitHub
  if you choose this option.
\end{itemize}

\noindent Both submission methods are graded using the same rubric.

\section*{Mini codebook (use this; do not guess)}

Each row represents one movie. Relevant variables include:
\begin{itemize}
  \item \texttt{budget}: Production budget in USD (0 if unavailable).
  \item \texttt{revenue}: Worldwide box office revenue in USD (0 if unavailable).
  \item \texttt{director}: Director of the movie.
  \item \texttt{runtime}: Movie length in minutes.
  \item \texttt{vote\_average}: Average user rating (0--10).
  \item \texttt{vote\_count}: Number of user votes.
  \item \texttt{popularity}: Popularity score based on user engagement.
\end{itemize}

\section*{Questions}

\begin{enumerate}

  \item \textbf{Get the movie dataset (proof required).}
  \begin{enumerate}
    \item Choose \textbf{one} method:
    \begin{itemize}
      \item \textbf{GitHub option:} In the \textbf{Terminal tab}, run:
\begin{verbatim}
git status
git pull
\end{verbatim}
      \item \textbf{Canvas option:} Download the movie dataset from Canvas and place it in your project \texttt{data/} folder.
    \end{itemize}

    \item \textbf{Pseudo-code (follow structure; fill in details):}
\begin{verbatim}
# confirm working directory
__________()

# list files in data directory
__________("data")

# visually confirm expected movie file exists
\end{verbatim}

    \item Confirm the movie dataset exists in your project.
    \item \textbf{Proof (write-up):} In \texttt{outputs/writeup.md}, paste:
    \begin{itemize}
      \item the output of \texttt{getwd()},
      \item the output of \texttt{list.files("data")}.
    \end{itemize}
  \end{enumerate}

  \item \textbf{Load and summarize the dataset.}
  \begin{enumerate}
    \item Load the movie dataset into an object named \texttt{df}.

    \item \textbf{Pseudo-code (intentionally incomplete):}
\begin{verbatim}
df <- read.csv("data/__________.csv")

# quick structure checks (choose at least two)
__________(df)
__________(df)
__________(df)
\end{verbatim}

    \item Summarize the dataset to understand its structure.

    \item \textbf{Proof (write-up):} Report:
    \begin{itemize}
      \item number of rows and columns,
      \item the range of \texttt{budget},
      \item the range of \texttt{revenue}.
    \end{itemize}
  \end{enumerate}

  \item \textbf{Plot distributions: movie budget and revenue.}

  Create two histograms:
  \begin{itemize}
    \item one for \texttt{budget},
    \item one for \texttt{revenue}.
    \item Repeat the process again and chose alternative bin sizes to assess how that changes the interpretation. Use one of the following formulas: 
    
    \begin{itemize}
  \item \textbf{Freedman--Diaconis (bin width):}
  \[
    h = 2 \cdot \mathrm{IQR}(x) \cdot n^{-1/3}
  \]
  \item \textbf{Scott's Rule (bin width):}
  \[
    h = 3.5 \cdot s(x) \cdot n^{-1/3}
  \]
  \item \textbf{Sturges' Rule (number of bins):}
  \[
    k = \lceil \log_2(n) + 1 \rceil
  \]
\end{itemize}

  \end{itemize}

  \item \textbf{Pseudo-code (structure only):}
\begin{verbatim}
# budget histogram
ggplot(df, aes(x = ________)) +
  geom_histogram(__________)

# revenue histogram
ggplot(df, aes(x = ________)) +
  geom_histogram(__________)
\end{verbatim}

  \noindent Save the plots as:
  \begin{itemize}
    \item \texttt{figures/budget\_hist.png}
    \item \texttt{figures/revenue\_hist.png}
  \end{itemize}

  \item \textbf{Identify top-grossing directors and compare revenue.}
  \begin{enumerate}
    \item Identify the \textbf{top three directors} by total box office revenue.
    \item Subset the data to movies directed by these three directors.
    \item Create a \textbf{boxplot} showing the distribution of \texttt{revenue} for each director.
  \end{enumerate}

  \item \textbf{Pseudo-code (leave blanks):}
\begin{verbatim}
top_directors <- df %>%
  group_by(__________) %>%
  summarize(total_revenue = ________) %>%
  arrange(__________) %>%
  slice(____)

df_top <- df %>%
  filter(__________ %in% top_directors$__________)
\end{verbatim}

  \noindent Save the plot as:
  \[
    \texttt{figures/revenue\_by\_director.png}
  \]

  \item \textbf{Scatter plot with size and color encodings.}

  Create a scatter plot with:
  \begin{itemize}
    \item x-axis: \texttt{budget}
    \item y-axis: \texttt{revenue}
  \end{itemize}

  Then:
  \begin{itemize}
    \item map one quantitative variable to \textbf{point size},
    \item map one categorical variable to \textbf{color}.
  \end{itemize}

  \item \textbf{Pseudo-code (structure only):}
\begin{verbatim}
ggplot(df, aes(x = ________, y = ________,
               size = ________, color = ________)) +
  geom_point(__________)
\end{verbatim}

  \noindent Save the plot as:
  \[
    \texttt{figures/budget\_revenue\_scatter.png}
  \]

  \item \textbf{Interpretation (write-up required).}

  In \texttt{outputs/writeup.md}, write 10--14 sentences addressing:
  \begin{itemize}
    \item what the budget and revenue distributions reveal,
    \item how revenues differ across top directors,
    \item what relationships are most salient in the scatter plot,
    \item how size and color encodings affect interpretation.
  \end{itemize}

  \item \textbf{Submit your work (GitHub \emph{or} Canvas) + proof required.}
  \begin{enumerate}
    \item \textbf{Choose ONE submission method:}
    \begin{itemize}
      \item \textbf{GitHub option:} Commit and push your work to GitHub.
      \item \textbf{Canvas option:} Upload \texttt{scripts/lab.R}, \texttt{outputs/writeup.md}, and all required figures to Canvas.
    \end{itemize}

    \item \textbf{Pseudo-code (submission skeleton):}
\begin{verbatim}
# GitHub option
git status
git add ______
git commit -m "__________"
git push
\end{verbatim}

    \item \textbf{Proof (write-up):}
    \begin{itemize}
      \item If using \textbf{GitHub}: paste \texttt{git status} and \texttt{git log -1}.
      \item If using \textbf{Canvas}: paste \texttt{list.files("scripts")},
            \texttt{list.files("outputs")}, and \texttt{list.files("figures")},
            and state that you submitted via Canvas.
    \end{itemize}
  \end{enumerate}

\end{enumerate}

\section*{Optional challenge (if you finish early)}
Choose one plot and create an alternative version optimized for a \textbf{general public} audience. In 5--7 sentences, explain what design choices you changed and why.

\section*{Checklist (before you leave)}
\begin{itemize}
  \item \texttt{scripts/lab.R} runs top-to-bottom
  \item \texttt{outputs/writeup.md} exists and includes interpretation + proofs
  \item Required figures exist in \texttt{figures/}
  \item Work is either committed and pushed to GitHub \emph{or} uploaded to Canvas
\end{itemize}

\end{document}
