\documentclass[11pt]{article}
\usepackage{amsmath, amssymb, graphicx, hyperref}
\usepackage{enumitem}
\setlist{nosep}
\usepackage[margin=1in]{geometry}

\title{In-Class Problem Set: Reproducible Visualization Workflow (R + GitHub)}
\author{}
\date{}

\begin{document}
\maketitle

\noindent \textbf{Goal.} Practice a simple, reproducible visualization workflow in R:
load a dataset, make a small number of plots using fixed variables, save your output,
and push your work to GitHub.

\medskip
\noindent \textbf{What to submit (in your GitHub repo).}
\begin{itemize}
  \item A script file: \texttt{scripts/lab.R}
  \item A short write-up: \texttt{outputs/writeup.md}
  \item Saved figures: at least 3 image files in \texttt{figures/}
\end{itemize}

\medskip
\noindent \textbf{Rules.}
\begin{itemize}
  \item Work inside an \textbf{single R Project}.
  \item Use a \textbf{sequential, hard-coded workflow} (no user-defined functions).
  \item You may consult notes and documentation. If you use external code, cite it.
\end{itemize}

\section*{Questions}

\begin{enumerate}

%-------------------------------------------------
\item \textbf{Create an R Project (proof required).}
\begin{enumerate}
  \item Create an R Project for this course.
  \item \textbf{Proof:} In \texttt{outputs/writeup.md}, include:
  \begin{itemize}
    \item the output of \texttt{getwd()}, and
    \item a screenshot showing the project name in RStudio.
  \end{itemize}
\end{enumerate}

%-------------------------------------------------
\item \textbf{Load the dataset.}
\begin{enumerate}
  \item Confirm the dataset file exists in \texttt{data/}.
  \item Load the dataset into R as an object named \texttt{vdem}.

\begin{verbatim}
# Step 1: load the dataset
vdem <- readRDS("data/vdem.rds")
\end{verbatim}

  \item \textbf{Proof:} In \texttt{outputs/writeup.md}, include:
  \begin{itemize}
    \item the dimensions of \texttt{vdem} (rows $\times$ columns),
    \item the first three column names.
  \end{itemize}

\begin{verbatim}
# Proof code
dim(vdem)
names(vdem)[1:3]
\end{verbatim}

\end{enumerate}

%-------------------------------------------------
\item \textbf{Variables used in all plots.}

For all plots in this assignment, use the following variables:
\begin{itemize}
  \item \texttt{v2clacjstw} (access to justice for women)
  \item \texttt{v2clacjstm} (access to justice for men)
  \item \texttt{v2clkill} (freedom from political killings)
  \item \texttt{v2cltort} (freedom from torture)
\end{itemize}

Treat all variables as continuous.

\medskip
\textbf{Proof:} In \texttt{outputs/writeup.md}, show that these variables exist and report their class.

\begin{verbatim}
names(vdem)
str(vdem[, c("v2clacjstw","v2clacjstm","v2clkill","v2cltort")])
\end{verbatim}

%-------------------------------------------------
\item \textbf{Create the baseline plot.}

Create a scatterplot with:
\begin{itemize}
  \item x-axis: \texttt{v2clacjstw}
  \item y-axis: \texttt{v2clacjstm}
\end{itemize}

\begin{verbatim}
library(ggplot2)

p0 <- ggplot(vdem, aes(
  x = v2clacjstw,
  y = v2clacjstm
)) +
  geom_point()

p0
\end{verbatim}

Save the plot to \texttt{figures/}.

\begin{verbatim}
ggsave("figures/plot_baseline.png", plot = p0)
\end{verbatim}

%-------------------------------------------------
\item \textbf{Create two plot extensions (your design choices).}

Starting from the baseline plot, create \textbf{two} additional figures.
For each extension, you will make a \textbf{design choice} about how to encode extra information.

\medskip
\textbf{Extension 1 (your choice: add ONE encoding).}

Create a version of the baseline plot where you add \textbf{one} of the following:
\begin{itemize}
  \item map one variable to \textbf{color}, \emph{or}
  \item map one variable to \textbf{size}.
\end{itemize}

You must use one of these variables as your added information:
\texttt{v2clkill} or \texttt{v2cltort}.

\begin{verbatim}
# PSEUDOCODE:
# Choose ONE:
# - color = <your choice: v2clkill OR v2cltort>
# OR
# - size  = <your choice: v2clkill OR v2cltort>

p1 <- ggplot(vdem, aes(
  x = v2clacjstw,
  y = v2clacjstm,
  <your choice goes here>
)) +
  geom_point()

p1
\end{verbatim}

Save this plot.

\begin{verbatim}
ggsave("figures/plot_extension1.png", plot = p1)
\end{verbatim}

\medskip
\textbf{Extension 2 (your choice: add a SECOND encoding).}

Create another version where you add a \textbf{second} encoding, so your plot uses \textbf{both}
color and size. Use the remaining variable that you did \emph{not} use in Extension 1.

\begin{verbatim}
# PSEUDOCODE:
# If Extension 1 used:
#   color = v2clkill
# then Extension 2 should add:
#   size  = v2cltort
#
# If Extension 1 used:
#   size  = v2cltort
# then Extension 2 should add:
#   color = v2clkill

p2 <- ggplot(vdem, aes(
  x = v2clacjstw,
  y = v2clacjstm,
  <your two choices go here>
)) +
  geom_point()

p2
\end{verbatim}

Save this plot.

\begin{verbatim}
ggsave("figures/plot_extension2.png", plot = p2)
\end{verbatim}

%-------------------------------------------------
\item \textbf{Brief written interpretation.}

In \texttt{outputs/writeup.md}, write \textbf{3 short bullets}:
\begin{itemize}
  \item One thing you can see in the baseline plot.
  \item One thing your Extension 1 makes easier to see.
  \item One thing your Extension 2 makes easier (or harder) to see.
  \item What argument is being conveyed by these plots. What are some ways the argument can be improved?
\end{itemize}

%-------------------------------------------------
\item \textbf{GitHub submission (proof required).}

Commit and push your work.

\begin{verbatim}
git add .
git commit -m "Finish in-class visualization lab"
git push
\end{verbatim}

\textbf{Proof:} In \texttt{outputs/writeup.md}, include:
\begin{itemize}
  \item the output of \texttt{git status} after committing,
  \item the commit hash or a screenshot of the GitHub repo.
\end{itemize}

\end{enumerate}

\section*{Checklist (before you leave)}
\begin{itemize}
  \item \texttt{scripts/lab.R} runs top-to-bottom without errors
  \item \texttt{outputs/writeup.md} includes all required proofs
  \item Three figures saved in \texttt{figures/}
  \item Changes pushed to GitHub
\end{itemize}

\end{document}
