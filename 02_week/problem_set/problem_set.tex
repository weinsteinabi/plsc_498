\documentclass[11pt]{article}
\usepackage{amsmath, amssymb, graphicx, hyperref}
\usepackage{enumitem}
\setlist{nosep}
\usepackage[margin=1in]{geometry}

\title{In-Class Problem Set: Reproducible Visualization Workflow (R + GitHub)}
\author{}
\date{}

\begin{document}
\maketitle

\noindent \textbf{Goal.} Extend the code from the lecture slides to produce a small, reproducible workflow: load the provided dataset, create multiple figures using explicit mappings, and submit your work through GitHub.

\medskip
\noindent \textbf{What to submit (in your GitHub repo).}
\begin{itemize}
  \item A script file: \texttt{scripts/lab.R}
  \item A short write-up: \texttt{outputs/writeup.md}
  \item Saved figures: at least 4 image files in \texttt{figures/}
  \item (Optional but recommended) a log file: \texttt{outputs/log.txt}
\end{itemize}

\medskip
\noindent \textbf{Rules.}
\begin{itemize}
  \item Work inside an \textbf{R Project}.
  \item Use a \textbf{sequential, hard-coded workflow} (no user-defined functions).
  \item You may consult notes and documentation. If you use any external code, cite it in your write-up.
\end{itemize}

\section*{Questions}

\begin{enumerate}

  \item \textbf{Create an R Project (proof required).}
  \begin{enumerate}
    \item Create an R Project for this course on your computer.
    \item \textbf{Proof:} In your \texttt{outputs/writeup.md}, include:
    \begin{itemize}
      \item the output of \texttt{getwd()} run from inside the project, and
      \item a screenshot showing the \texttt{.Rproj} file in your project folder \emph{or} the RStudio Project name visible in the RStudio window.
    \end{itemize}
  \end{enumerate}

  \item \textbf{Load the provided dataset from the \texttt{data/} folder.}
  \begin{enumerate}
    \item Confirm the dataset file exists in \texttt{data/}. (Do not manually move it.)
    \item Write code in \texttt{scripts/lab.R} to load it into R as an object named \texttt{vdem}.
\begin{verbatim}
# PSEUDOCODE:
# 1) read the .rds file from the data/ folder
# 2) store it as vdem

vdem <- readRDS("data/vdem.rds")
\end{verbatim}

    \item \textbf{Proof:} In \texttt{outputs/writeup.md}, include:
    \begin{itemize}
      \item the dimensions of \texttt{vdem} (rows $\times$ columns), and
      \item the first 3 column names.
    \end{itemize}

\begin{verbatim}
# PSEUDOCODE for the proof:
dim(vdem)
names(vdem)[1:3]
\end{verbatim}

  \end{enumerate}

  \item \textbf{Work with fixed variables (no selection required).}
  \begin{enumerate}
    \item For all plots in this assignment, use the following variables from the dataset:
    \begin{itemize}
      \item Access to justice for women: \texttt{v2clacjstw}
      \item Access to justice for men: \texttt{v2clacjstm}
      \item Freedom from political killings: \texttt{v2clkill}
      \item Freedom from torture: \texttt{v2cltort}
    \end{itemize}

    \item Treat \emph{all four variables as continuous}. You do \textbf{not} need to identify or justify variable types.

    \item Your baseline plot will use the following mapping (note: no trailing comma):
\begin{verbatim}
# PSEUDOCODE:
# x = justice for women
# y = justice for men
# points = one row per observation

library(ggplot2)

p0 <- ggplot(vdem, aes(x = v2clacjstw, y = v2clacjstm)) +
  geom_point()

p0
\end{verbatim}

    \item Then create a version that uses \textbf{both} color and size:
\begin{verbatim}
# PSEUDOCODE:
# color = freedom from political killings
# size  = freedom from torture

p1 <- ggplot(vdem, aes(
  x = v2clacjstw,
  y = v2clacjstm,
  color = v2clkill,
  size  = v2cltort
)) +
  geom_point()

p1
\end{verbatim}

    \item \textbf{Proof:} In \texttt{outputs/writeup.md}, include:
    \begin{itemize}
      \item confirmation that all four variables exist in the dataset (show \texttt{names(vdem)} output or a short snippet), and
      \item their reported class from \texttt{str()}.
    \end{itemize}

\begin{verbatim}
# PSEUDOCODE for the proof:
names(vdem)
str(vdem[, c("v2clacjstw","v2clacjstm","v2clkill","v2cltort")])
\end{verbatim}

  \end{enumerate}

\medskip
\textbf{Pseudocode guide (follow this order in \texttt{scripts/lab.R}):}
\begin{enumerate}
  \item Load required libraries (\texttt{ggplot2}, optionally \texttt{tidyverse}).
  \item Read the dataset from \texttt{data/} into \texttt{vdem}.
  \item Confirm required variables exist (\texttt{names(vdem)}; \texttt{str(...)}).
  \item Make the baseline plot (x = \texttt{v2clacjstw}, y = \texttt{v2clacjstm}).
  \item Make the enhanced plot (add color = \texttt{v2clkill}, size = \texttt{v2cltort}).
  \item Save plots with \texttt{ggsave()} into \texttt{figures/}.
\end{enumerate}

  \item \textbf{Create a reproducible folder structure + (optional) logging.}
  \begin{enumerate}
    \item Ensure these folders exist in your project:
    \begin{itemize}
      \item \texttt{scripts/}
      \item \texttt{outputs/}
      \item \texttt{figures/}
      \item \texttt{logs/} \textit{(optional but recommended)}
    \end{itemize}
    \item In \texttt{scripts/lab.R}, add code that creates any missing directories (without errors).
\begin{verbatim}
# PSEUDOCODE:
# if a folder does not exist, create it
dir.create("scripts", showWarnings = FALSE)
dir.create("outputs", showWarnings = FALSE)
dir.create("figures", showWarnings = FALSE)
dir.create("logs", showWarnings = FALSE)
\end{verbatim}
    \item \textbf{Proof:} In \texttt{outputs/writeup.md}, include \texttt{list.files()} output showing the folders.
\begin{verbatim}
# PSEUDOCODE for proof:
list.files()
\end{verbatim}

    \item \textbf{Optional (challenge):} Create a simple log file \texttt{outputs/log.txt} that records:
    \begin{itemize}
      \item the current date/time,
      \item the dataset filename loaded,
      \item and the names of the four required variables.
    \end{itemize}
  \end{enumerate}

  \item \textbf{Make three plot extensions + comment on them.}

  Using your baseline plot as the starting point, create \textbf{three} distinct extensions (three separate figures). Each figure must include a caption in your write-up that explains:
  \begin{itemize}
    \item what variables are mapped to what visual properties,
    \item what comparison is easiest to make,
    \item and one default choice you are accepting (or changing) and why.
  \end{itemize}

  \medskip
  Your three extensions must come from different categories below (choose any three):
  \begin{enumerate}
    \item \textbf{Add an annotation layer:} add a title + axis labels.
    \item \textbf{Handle overplotting:} use transparency (\texttt{alpha}) and briefly explain why.
    \item \textbf{Scale adjustment:} adjust the color scale and/or size scale and explain the effect.
  \end{enumerate}

  \medskip
  \textbf{Saving requirement:} Save each plot to \texttt{figures/} using \texttt{ggsave()} (do not rely on screenshots). Name files clearly (e.g., \texttt{figures/plot1.png}, \texttt{figures/plot2.png}, \texttt{figures/plot3.png}).

\begin{verbatim}
# PSEUDOCODE for saving:
ggsave("figures/plot_baseline.png", plot = p0, width = 7, height = 5)
ggsave("figures/plot_color_size.png", plot = p1, width = 7, height = 5)
\end{verbatim}

  \item \textbf{If you finish early:}
  \begin{itemize}
    \item Add a short ``limitations'' note (1--2 sentences) about what the plot cannot show.
    \item Add a deliberately ``bad'' version and write 3 bullets on why it misleads.
  \end{itemize}

  \item \textbf{GitHub requirement:} Commit and push your work.

  \textbf{Proof:} In \texttt{outputs/writeup.md}, include:
  \begin{itemize}
    \item the output of \texttt{git status} \emph{after} committing (showing a clean working tree), and
    \item either a screenshot of your GitHub repo showing the latest commit \emph{or} the commit hash and message.
  \end{itemize}

\begin{verbatim}
# PSEUDOCODE (terminal):
git add .
git commit -m "finish lab"
git status
git push
\end{verbatim}

\end{enumerate}

\section*{Checklist (before you leave)}
\begin{itemize}
  \item \texttt{scripts/lab.R} exists and runs top-to-bottom
  \item \texttt{outputs/writeup.md} exists and includes required proofs
  \item At least 4 figures saved in \texttt{figures/}
  \item Work is pushed to GitHub
\end{itemize}

\end{document}
