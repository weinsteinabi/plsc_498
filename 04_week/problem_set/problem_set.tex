\documentclass[11pt]{article}
\usepackage{amsmath, amssymb, graphicx, hyperref}
\usepackage{enumitem}
\setlist{nosep}
\usepackage[margin=1in]{geometry}

\title{In-Class Problem Set: Color Encodings with Senate Ideology Data (R + GitHub \emph{or} Canvas)}
\author{}
\date{}

\begin{document}
\maketitle

\noindent \textbf{Goal.} Practice using color intentionally (not decoratively) by visualizing ideology in the U.S.\ Senate across four time points. You will (i) obtain the data from the course materials (GitHub \emph{or} Canvas), (ii) subset to the target Senates, (iii) make one plot for each time point with color encoding ideology, (iv) interpret what you see, and (v) submit via \textbf{GitHub \emph{or} Canvas}.

\medskip
\noindent \textbf{What to submit (GitHub \emph{or} Canvas).}
\begin{itemize}
  \item A script file: \texttt{scripts/lab.R}
  \item A short write-up: \texttt{outputs/writeup.md}
  \item Four figures (one per time point) saved to \texttt{figures/}
\end{itemize}

\noindent If you submit via Canvas, upload the same files listed above as individual files (or upload a single zip that preserves the folder structure).

\medskip
\noindent \textbf{Rules (read carefully).}
\begin{itemize}
  \item Work inside an \textbf{R Project}.
  \item Use a \textbf{sequential, hard-coded workflow} (no user-defined functions).
  \item Your plots must:
  \begin{itemize}
    \item set a non-default plot background color,
    \item use colors that are intuitive for the task,
    \item use an accessibility-conscious palette (colorblind-friendly),
    \item and \textbf{explicitly justify your color scale choice} (sequential vs diverging) in your write-up.
  \end{itemize}
  \item Save outputs using \texttt{ggsave()} (no screenshots).
  \item If you choose the \textbf{GitHub submission option}, Git commands go in the \textbf{Terminal tab} (not the R Console).
\end{itemize}

\section*{Submission options}

You may submit this assignment using \textbf{either} of the following methods:

\begin{itemize}
  \item \textbf{GitHub submission (recommended):} Commit and push your work to your GitHub repository. You will include Git proof (\texttt{git status} and \texttt{git log}) in your write-up.
  \item \textbf{Canvas submission:} Upload the required files directly to Canvas. You do \emph{not} need to use GitHub if you choose this option.
\end{itemize}

\noindent Both submission methods are graded using the same rubric.

\section*{Mini codebook (use this; do not guess)}

\begin{itemize}
  \item \textbf{Which Senates to use.} For this problem set, use the following Congress numbers to represent the four target time points:
  \begin{itemize}
    \item \textbf{1990} $\rightarrow$ \textbf{101st Congress}
    \item \textbf{2000} $\rightarrow$ \textbf{106th Congress}
    \item \textbf{2010} $\rightarrow$ \textbf{111th Congress}
    \item \textbf{2020} $\rightarrow$ \textbf{116th Congress}
  \end{itemize}
  \item \textbf{What DW-NOMINATE is.} DW-NOMINATE is an ideology scaling procedure based on roll-call voting. In most datasets:
  \begin{itemize}
    \item \texttt{dwnom1} is the primary (left--right) ideological dimension,
    \item \texttt{dwnom2} is a secondary ideological dimension (often smaller and context-dependent).
  \end{itemize}
  \item \textbf{What you will plot.} Your scatterplots should use the two ideology dimensions (typically \texttt{dwnom1} on x and \texttt{dwnom2} on y), and \textbf{color points by ideology} (typically \texttt{dwnom1}) unless your course codebook specifies a different ideology column.
  \item \textbf{Color scale rule (required).}
  \begin{itemize}
    \item If the ideology variable has meaningful sign around 0 (e.g., negative vs positive), use a \textbf{diverging} scale centered at 0.
    \item If you treat ideology as magnitude only (no meaningful center), use a \textbf{sequential} scale.
  \end{itemize}
  In your write-up, state which rule you used and why.
\end{itemize}

\section*{Questions}

\begin{enumerate}

  \item \textbf{Get the Senate ideology data (proof required).}
  \begin{enumerate}
    \item Choose \textbf{one} method:
    \begin{itemize}
      \item \textbf{GitHub option:} In the \textbf{Terminal tab}, run:
\begin{verbatim}
git status
git pull
\end{verbatim}
      \item \textbf{Canvas option:} Download the Senate ideology file from Canvas and place it in your project \texttt{data/} folder.
    \end{itemize}

    \item Confirm the Senate ideology file exists in your project (the exact file name/path is in the course materials; for most of you it will be in \texttt{data/}).

    \item \textbf{Pseudo-code (do not make this perfect; fill in blanks):}
\begin{verbatim}
# confirm you are in the project root
__________()

# list files in the data folder
__________("data")

# (optional) search for the Senate file name pattern
# _________("data", pattern="_____")
\end{verbatim}

    \item \textbf{Proof (write-up):} In \texttt{outputs/writeup.md}, paste:
    \begin{itemize}
      \item the output of \texttt{getwd()} from inside your R Project, and
      \item the output of \texttt{list.files("data")} showing the Senate file.
    \end{itemize}
  \end{enumerate}

  \item \textbf{Set up your reproducible workflow (folders + script).}
  \begin{enumerate}
    \item Ensure your project contains these folders (create them if missing):
    \begin{itemize}
      \item \texttt{scripts/}
      \item \texttt{outputs/}
      \item \texttt{figures/}
    \end{itemize}

    \item Create a script named \texttt{scripts/lab.R}. All code for this problem set must live in this script.

    \item \textbf{Pseudo-code (structure only):}
\begin{verbatim}
# from the project root:
__________("scripts")
__________("outputs")
__________("figures")

# confirm what is in the root folder
__________()
\end{verbatim}

    \item \textbf{Suggested edit (important):} At the top of \texttt{scripts/lab.R}, include:
    \begin{itemize}
      \item a short header comment describing what the script does,
      \item \texttt{library(...)} calls,
      \item \texttt{set.seed(123)}.
    \end{itemize}

    \item \textbf{Pseudo-code (script header):}
\begin{verbatim}
# ----------------------------------
# Script name: __________
# Purpose: ________________________
# ----------------------------------

library(__________)
library(__________)

set.seed(_____)
\end{verbatim}

    \item \textbf{Proof (write-up):} paste the output of \texttt{list.files()} from your project root.
  \end{enumerate}

  \item \textbf{Load and summarize the dataset.}
  \begin{enumerate}
    \item In \texttt{scripts/lab.R}, load the dataset into an object named \texttt{df}.

    \item \textbf{Pseudo-code (intentionally incomplete):}
\begin{verbatim}
# load data (fill in filename)
df <- read.csv("data/__________.csv")

# quick checks (pick at least two)
dim(df)
names(df)
head(df)

# check the ideology columns exist
# summary(df$______)
# summary(df$______)
\end{verbatim}

    \item Summarize the dataset in a way that supports your next steps. At minimum include:
    \begin{itemize}
      \item \texttt{dim(df)}
      \item \texttt{names(df)}
      \item a focused summary of the key columns you will use (time/Congress, chamber, ideology).
    \end{itemize}

    \item \textbf{Proof (write-up):} Report:
    \begin{itemize}
      \item number of rows and columns,
      \item the column you will use for Congress/time,
      \item the column you will use to identify the Senate chamber (if applicable),
      \item the ideology columns you will use (e.g., \texttt{dwnom1}, \texttt{dwnom2}).
    \end{itemize}
  \end{enumerate}

  \item \textbf{Subset to the four target Senates (required).}
  \begin{enumerate}
    \item Subset the data so it contains only \textbf{Senate} observations for the following Congresses:
    \[
      \{101, 106, 111, 116\}.
    \]
    \item Save the subset as \texttt{df4}.

    \item \textbf{Pseudo-code (leave blanks):}
\begin{verbatim}
# choose the Congress variable name
# unique(df$_______)

df4 <- df %>%
  filter(_________ == "Senate") %>%     # or the correct chamber coding
  filter(_________ %in% c(101, 106, ____, ____))

# quick checks
table(df4$_________)      # Congress counts
# table(df4$_________)    # chamber counts (if applicable)
\end{verbatim}

    \item \textbf{Proof (write-up):} Include counts that confirm:
    \begin{itemize}
      \item only these four Congresses appear in \texttt{df4}, and
      \item \texttt{df4} contains only Senate observations (not House).
    \end{itemize}
  \end{enumerate}

  \item \textbf{Make four plots (one per Congress), with color encoding ideology (required).}

  For each of the four Congresses, create a scatterplot using the two DW-NOMINATE dimensions:
  \begin{itemize}
    \item x-axis: \texttt{dwnom1}
    \item y-axis: \texttt{dwnom2}
    \item color: ideology (typically \texttt{dwnom1})
  \end{itemize}

  \noindent \textbf{Required design constraints (integrated).} Each plot must:
  \begin{itemize}
    \item set a non-default plot background color,
    \item use an accessibility-conscious palette,
    \item use a color scale that is \textbf{intuitive for the task}:
      \begin{itemize}
        \item \textbf{diverging centered at 0} if ideology sign matters, or
        \item \textbf{sequential} if you treat ideology as magnitude only,
      \end{itemize}
    \item label axes and the legend clearly (what the variable is).
  \end{itemize}

  \noindent Save figures to \texttt{figures/} with clear names, for example:
  \begin{itemize}
    \item \texttt{figures/senate\_101.png}
    \item \texttt{figures/senate\_106.png}
    \item \texttt{figures/senate\_111.png}
    \item \texttt{figures/senate\_116.png}
  \end{itemize}

  \medskip
  \noindent \textbf{Pseudo-code (intentionally incomplete; do not perfect this):}
\begin{verbatim}
# pick ONE Congress first, then repeat for others
# congress_id <- ____

p <- ggplot(df4 %>% filter(_________ == ____),
            aes(x = ________, y = ________, color = ________)) +
  geom_point(__________) +
  labs(title = "Senate ideology: ____ Congress",
       x = "________", y = "________", color = "________") +
  theme_minimal() +
  theme(plot.background = element_rect(fill = "________", color = NA))

# choose a palette + scale type (diverging or sequential)
# scale_color___________(...)

ggsave("figures/senate____.png", p, width = ____, height = ____)
\end{verbatim}

  \item \textbf{Interpretation (write-up required).}

  In \texttt{outputs/writeup.md}, write 10--14 sentences answering:
  \begin{itemize}
    \item What does color represent in your plots (which variable, which direction, what range)?
    \item Compare the earliest vs latest Congress in your set: what changed in ideological separation and dispersion?
    \item \textbf{Color-scale justification (required):} Did you use a sequential or diverging scale? Why does that choice match the meaning of ideology in your plot?
    \item Accessibility: state one concrete decision you made to improve accessibility (palette choice, contrast with background, labeling).
  \end{itemize}
  
  
  \item Choose \textbf{one} of your four Senate plots to evaluate.

  \begin{itemize}
    \item Upload the image file to a color-blindness testing website.
  You may use any reputable tool, such as: \url{https://www.color-blindness.com/coblis-color-blindness-simulator/}, or another simulator discussed in class
  \item  Examine how your plot appears under at least \textbf{two} simulated conditions
  (e.g., deuteranopia, protanopia, tritanopia).
  \item If the plot is \textbf{not} clearly readable under at least one simulation,
  revise your color scale and re-export the plot.
  You may:
  \begin{itemize}
    \item change the palette,
    \item adjust the midpoint or limits of the scale,
    \item change the background color,
    \item or adjust point size / opacity.
  \end{itemize}
   \item Save the final version (even if unchanged) using the \emph{same filename}
  so it overwrites the original plot.
\item \textbf{Write-up (required):} In \texttt{outputs/writeup.md}, write 6--9 sentences describing:
  \begin{itemize}
    \item which plot you tested and which simulator you used,
    \item which color-vision deficiencies you examined,
    \item what problems (if any) you observed,
    \item what changes you made \emph{or} why you decided no change was necessary,
    \item and one general lesson about color accessibility you learned from this process.
  \end{itemize}
  \end{itemize}

  \item \textbf{Submit your work (GitHub \emph{or} Canvas) + proof required.}
  \begin{enumerate}
    \item \textbf{Choose ONE submission method:}
    \begin{itemize}
      \item \textbf{GitHub option:} Commit and push your work to GitHub.
      \item \textbf{Canvas option:} Upload \texttt{scripts/lab.R}, \texttt{outputs/writeup.md}, and the four figure files from \texttt{figures/} to Canvas.
    \end{itemize}

    \item \textbf{Pseudo-code (submission skeleton; fill in blanks):}
\begin{verbatim}
# GitHub option (Terminal tab)
git status
git add ______
git commit -m "__________"
git push
\end{verbatim}

    \item \textbf{Proof (write-up):}
    \begin{itemize}
      \item If using \textbf{GitHub}: paste \texttt{git status} (clean working tree) and \texttt{git log -1}.
      \item If using \textbf{Canvas}: paste \texttt{list.files("scripts")}, \texttt{list.files("outputs")},  \\
      \texttt{list.files("figures")}, and write one sentence stating you submitted via Canvas.
    \end{itemize}
  \end{enumerate}

\end{enumerate}

\section*{Optional challenge (if you finish early)}
Create a second version of one Congress plot that changes exactly \textbf{one} design element:
\begin{itemize}
  \item Switch sequential $\leftrightarrow$ diverging (and explain why the alternative is worse), \emph{or}
  \item Keep the same palette but change the background color and explain how contrast changes readability.
\end{itemize}
In 5--7 sentences, argue which version is better for (i) a general audience and (ii) an expert audience.

\section*{Checklist (before you leave)}
\begin{itemize}
  \item \texttt{scripts/lab.R} exists and runs top-to-bottom
  \item \texttt{outputs/writeup.md} exists and includes interpretation + proofs
  \item Four figures saved in \texttt{figures/} (101, 106, 111, 116)
  \item Work is either committed and pushed to GitHub \emph{or} uploaded to Canvas
\end{itemize}

\end{document}

